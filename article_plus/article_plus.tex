% Title:
% 	ARTICLE PLUS
% ----------------------
% Description:
% 	A template for scientific reports/articles.
%	Most necessary packages are imported,
%	so this should be a good starting point.
%
% Creator: Tommy O.

% ----------------------
% Package imports
% ----------------------
\documentclass[12pt, a4paper]{article}% 'twoside' for printing
\usepackage[utf8]{inputenc}% Allow input to be UTF-8
\usepackage[english]{babel}% Alternative: 'norsk'
\usepackage{graphicx}% For importing graphics
\usepackage{mathtools}% Fixes a few AMS bugs
\usepackage{amsthm, amsfonts, amssymb}% All the AMS packages
\usepackage{hyperref}% For \href{URL}{text}
\usepackage{fancyhdr}% For fancy headers
\usepackage[sharp]{easylist}% Easy nested lists
\usepackage{parskip}% Web-like paragraphs
%\usepackage{geometry}% May be used to set margins

% ----------------------
% Package setup
% ----------------------

% Theorems, definition and examples
\theoremstyle{plain} 
\newtheorem{theorem}{Theorem} 
\theoremstyle{definition} 
\newtheorem{definition}{Definition} 
\newtheorem{example}{Example}

% Setup for the fancyhdr package
\rhead{\thepage}
\lhead{\nouppercase{\leftmark}}

% Section numbers in equations
\numberwithin{equation}{section} 

% ----------------------
% Misc settings
% ----------------------

% Make theorem and definition titles bold
\makeatletter
\def\th@plain{%
	\thm@notefont{}% same as heading font
	\itshape % body font
}
\def\th@definition{%
	\thm@notefont{}% same as heading font
	\normalfont % body font
}
\makeatother

% ----------------------
% Document variables
% ----------------------

\title{ARTICLE PLUS}
\author{Tommy O.}

% ----------------------
% Document start
% ----------------------

\begin{document}
\maketitle
\pagestyle{fancy}
\begin{abstract}
Lorem ipsum ...
\end{abstract}
\tableofcontents

% ----------------------
% Document content start
% ----------------------

\section{SECTION}
Hello world. We start with a definition:
\begin{definition}[Group]
	A group is a set $S$ along with an operation $\circ$
	with four axioms: identity, associativity, closure and inverse.
\end{definition}
\begin{theorem}[Pytagorean theorem]
	\label{thm:pyta}
	For a triangle with sides $a$, $b$ and $c$,
	it is true that $a^2 + b^2 = c^2$.
\end{theorem}
\begin{proof}
	Here comes the proof...
\end{proof}
Look closely at theorem \ref{thm:pyta}, in other words theorem \eqref{thm:pyta}.
Here is a list:
\begin{itemize}
	\item But if the rebellion is to be successful.
	\item not to  between two, it is necessary.
	\item think that there is no.

\end{itemize}
\pagebreak
Another list:
\begin{easylist}[enumerate]
	# think that there is no.
	# When a rational conviction has.
	# in oneself whatever.
\end{easylist}

\begin{equation}
	T = \frac{1}{2}m v^2
\end{equation}

A .bib file will contain the bibliographic information of our document.

A .bib file will contain the bibliographic information of our document. I will only give a simple example, since there are many tools to generate the entries automatically.

A .bib file will contain the bibliographic information of our document. I will only give a simple example, since there are many tools to generate the entries automatically. I will not explain the structure of the file itself.

% -------------------------
% ---- BIBLIOGRAPHY
% -------------------------

\bibliographystyle{apalike} % 'alpha' is also good
%\bibliography{bibliography} % Reference to 'bibliography.bib'
\end{document}


