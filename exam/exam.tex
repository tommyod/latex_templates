% Title:
% 	EXAM TEMPLATE
% ----------------------
% Description:
% 	A template for exams/problem sheets.
%	For thorough documentation of the exam class,
%	see :
%	http://ftp.acc.umu.se/mirror/CTAN/macros/latex/contrib/exam/examdoc.pdf
%	http://www-math.mit.edu/~psh/exam/examdoc.pdf
%
% Creator: Tommy O.

% ------------------------------------------------------------------
% ---------------------- Package imports ---------------------------
% ------------------------------------------------------------------
\documentclass[12pt, addpoints, a4paper]{exam}
\usepackage[utf8]{inputenc}		% Allow input to be UTF-8
\usepackage[norsk]{babel}		% Alternative: 'norsk'
\usepackage{graphicx}			% For importing graphics
\usepackage{amsmath}			% AMS mathematics package
\usepackage{amsthm}				% AMS mathematics package
\usepackage{amsfonts}			% AMS mathematics package
\usepackage{amssymb}			% AMS mathematics package
\usepackage{mathtools}			% Fixes a few AMS bugs
\usepackage{hyperref}			% For \href{URL}{text}
\usepackage{enumitem}			% For enumeration
\usepackage{color} 				% Defines a new color
\usepackage{lastpage}			% For \pageref{LastPage}
\usepackage{booktabs}					% Professional tables
\usepackage{parskip}					% Paragraph space
\usepackage[sharp]{easylist}			% For easy lists
\usepackage[expansion=false]{microtype} % Fixes to make typography better
\usepackage[margin = 2.25cm, includehead, includefoot]{geometry}
%\usepackage[sc]{mathpazo}				% A nice font, alternative to CM

% ------------------------------------------------------------------
% ---------------------- Constants ---------------------------------
% ------------------------------------------------------------------
% The most important constants
\newcommand{\mytitle}{Exam title}
\newcommand{\mysubject}{R1 matematikk}
\newcommand{\mydate}{\today}
\newcommand{\myauthor}{Tommy O.}

% Redefine language stuff
\newcommand{\myrhead}{My school}
\newcommand{\mypagename}{Page}
\newcommand{\mycreated}{Created by:}
\newcommand{\mysolname}{Solution}
\renewcommand{\solutiontitle}{\noindent\textbf{\mysolname.}\hspace{0.75em}}
\pointpoints{point}{points}

% ------------------------------------------------------------------
% ---------------------- Settings ----------------------------------
% ------------------------------------------------------------------
%\shadedsolutions
\printanswers % Alternative: \noprintanswers, \printanswers
%\rhead{{\scshape {\footnotesize  \myrhead}}}
%\cfoot{\mypagename \enspace \thepage}
\definecolor{SolutionColor}{rgb}{0.8,0.9,1} % light blue

% Headers and footers
\runningfootrule
\firstpagefootrule
\firstpagefooter{\mysubject}{}{\mypagename\ \thepage\ av \pageref*{LastPage}}
\runningfooter{\mysubject}{}{\mypagename\ \thepage\ av \pageref*{LastPage}}
\runningheader{}{}{\includegraphics[width = 1.5cm]{figs/logo.pdf}}
\firstpageheadrule
\firstpageheader{\mydate}{\mytitle}{\pageref*{LastPage} sider totalt}

% Answer command for double lines
\def\answer#1{\underline{\underline{#1}}}

% ------------------------------------------------------------------
% ---------------------- Document start ----------------------------
% ------------------------------------------------------------------
\begin{document}
\pagestyle{headandfoot}
\noindent {\scshape \Large  \mytitle 
	\ifprintanswers
	\enspace (\mysolname	 )
	\fi
	} \\
\noindent {\mycreated \enspace  \myauthor} \vspace{1px}
\hrule \hrule
\subsection*{Instruks}
\begin{center}
\fbox{\fbox{\parbox{.9\textwidth}{
			\textbf{Del 1} \hspace*{1em} Innleveringen skal føres, ikke kladdes. Skriv \textbf{klart og tydelig} med penn. Vis nok utregninger til at jeg klart kan se hvordan du kom frem til svaret. Sett \answer{2 streker} under ditt endelige svar. Lever besvarelsen til meg i timen. \vspace*{1em} \\ 
	\textbf{Del 2} \hspace*{1em} Bruk Geogebra så mye du kan, ta skjermbilder og lim inn. Sett 2 streker under riktig svar og skriv nok til at jeg kan se hvordan du kom frem til svaret. Lever \texttt{.docx} og \texttt{.pdf} på ItsLearning.
}}}
\end{center}
\vspace{5mm}
%\noindent \makebox[\textwidth]{Fullt navn:\enspace\hrulefill}

% ------------------------------------------------------------------
% ---------------------- Content start -----------------------------
% ------------------------------------------------------------------
\section*{Del 1 -- Uten hjelpemidler}
\begin{questions}
\addpoints
\question {\bfseries Derivasjon}
\begin{parts}
\part[1] Deriver funksjonen $f(x) = \left( x^2 + 2\right)^4$
\begin{solution}
	Vi bruker kjerneregelen med $u = x^2 + 2$. Da blir $u' = 2x$ og vi får
	\begin{equation*}
		f'(x) = f'(u) u' = 4u^3 (2x) = \answer{8x \left(x^2 + 2\right)^3}
	\end{equation*}
\end{solution}

\part[1] Deriver funksjonen $g(x) = \frac{e^x}{x}$
\begin{solution}
	Skriv funksjonen som faktorer, slik at vi får $g(x) = \frac{e^x}{x} = e^x x^{-1}$.
	Bruk deretter produktregelen $(uv) = u'v + uv'$, med $u = e^x$ og $v = x^{-1}$.
	Vi får denne utregningen:
	\begin{align*}
	g'(x) &= \left( e^x \right)' x^{-1} + e^x \left( x^{-1} \right)' \\
	&= e^x x^{-1} + e^x (-1) x^{-2} \\
	&= \frac{e^x}{x} - \frac{e^x}{x^2} \\
	&= \frac{xe^x}{x^2} - \frac{e^x}{x^2} \\
	&= \answer{\frac{e^x(x-1)}{x^2}}
	\end{align*}
\end{solution}

\part[1] Deriver funksjonen $h(x) = e^{-x} \ln \left(x^2\right)$
\begin{solution}
	Vi må bruke kjernereglen og produktregelen. La $u = e^{-x}$ og $v = \ln \left(x^2\right)$, vi regner ut de deriverte
	\begin{align*}
		u' &= - e^{-x} \\
		v' &= \frac{1}{x^2} (2x) = \frac{2}{x}
	\end{align*}
	Nå bruker vi produktregelen $(uv) = u'v + uv'$
	\begin{align*}
	h'(x) &= u'v + uv' \\
		  &= - e^{-x} \ln \left(x^2\right)  + e^{-x}\frac{2}{x}  \\
		  &= - 2e^{-x} \ln \left(x\right)  + e^{-x}\frac{2}{x}  \\
		  &= \answer{2 e^{-x} \left(  \frac{1}{x} - \ln \left(x\right) \right) }
	\end{align*}
\end{solution}
\end{parts}
\end{questions}


\section*{Del 2 -- Med hjelpemidler}
\begin{questions}

\question {\bfseries Maksimere overskudd \\}
En bedrift har følgende totale kostnad og inntekt per dag knyttet til
produksjonen av varer, der $x$ er antall varer produsert på én dag.
\begin{align*}
	K(x) &= 0.1x^2 - 5x + 2200 \\
	I(x) &= 1200 \ln (x+1)
\end{align*}

\begin{parts}
\part[1] Bestem $K'(60)$ og $I'(60)$. Kan du ut i fra tallene si om bedriften
bør produsere flere eller færre enn 60 enheter per dag?
\begin{solution}
	Her bør du bruke CAS i Geogebra. Skriv inn \\
	\texttt{K(x) := 0.1*x*x - 5*x + 2200} \\
	\texttt{K'(60)} \\
	og du får at $\answer{K'(60) \approx 7}$. På samme måte får du at $\answer{I'(60) \approx 19.67}$.
	Ettersom grenseinntekten er høyere enn grensekostnaden bør bedriften
	produsere \answer{flere enn 60} enheter per dag.
\end{solution}

\part[1] Bestem produksjonsmengden som gir størst overskudd for bedriften.
\begin{solution}
	Definer en ny funksjon for overskuddet i Geogebra: \\
	\texttt{O(x) := I - K} \\
	Bruk så \\ \texttt{Ekstremalpunkt[ <Funksjon>, <Start>, <Slutt> ]}\\
	kommandoen til å finne maksimum.
	Vi får at $(x, y) = (90.54, 2853.08)$. Vi undersøker både $x = 90$ og $x = 91$.
	Når $x = 90$ får vi $O(90) = 2853.03$, og når $x = 91$ får vi $O(91) = 2853.05$.
	Overskuddet er størst når produksjonsmengden \answer{$x$ er lik $91$}
\end{solution}
\end{parts}



\end{questions}

%\hrule
%\subsection*{For retting}
%Ikke skriv noe her. \par \noindent
%\gradetable[h][questions]	
\end{document}
	
