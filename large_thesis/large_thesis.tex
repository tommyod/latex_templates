% Master thesis main document
% Author: Tommy O.
% Fall 2017

% -------------------------------------------------------------
% ---- DOCUMENT INITIALIZATION AND PACKAGE IMPORTS
% -------------------------------------------------------------

\documentclass[12pt, a4paper, twoside]{report} % 'twoside' when printing
\usepackage[utf8]{inputenc}				% UTF-8 input
\usepackage[english]{babel}				% Set language to english
\usepackage{blindtext}					% Use \Blinddocument or \blindmathpaper
\usepackage{graphicx}					% For graphics
\usepackage{fancyhdr}					% Fancy headers
\usepackage{hyperref}					% Internal and external hyperlinks
\usepackage{amsmath}					% Math from AMS
\usepackage{amsfonts}					% Fonts from AMS
\usepackage{amsthm}						% Theorems
\usepackage{amssymb}					% Symbols
\usepackage{enumitem}					% Enumeration
\usepackage{mathtools} 					% Bonus
\usepackage{color}						% Colors
\usepackage{booktabs}					% Professional tables
\usepackage{pdfpages}					% To include PDFs
\usepackage{parskip}					% Paragraph space
\usepackage{multicol}					% For multiple columns
\usepackage[sharp]{easylist}			% For easy lists
\usepackage{makeidx}					% For the index
\usepackage[linesnumbered,ruled]{algorithm2e}	% For algorithms
\usepackage{tikz-cd}					% For diagrams
\usepackage{listings}					% To include Python-code
\usepackage{etoolbox}					% To add symbol at end of examples
\usepackage[expansion=false]{microtype} % Fixes to make typography better
\usepackage[toc, page]{appendix} 		% To include appendices
%\usepackage[headings]{fullpage}		% Smaller margins
\usepackage[margin = 3cm, includehead, includefoot]{geometry}
%\usepackage[sc]{mathpazo}				% A nice font, alternative to CM
\usepackage{framed}						% To frame comments
\usepackage{multirow}					% For multiple rows in tables
\usepackage{afterpage}					% To insert blank pages
\usepackage{blindtext}					% To insert some blind text

% -------------------------------------------------------------
% ---- PACKAGE SET UP
% -------------------------------------------------------------

\graphicspath{{figs/}}					% Path to graphics
\pagestyle{fancy}						% Set page headers to fancy

% PACKAGE SET UP -- algorithm2e
\SetKw{Continue}{continue}
\SetKw{Break}{break}
\SetKw{Yield}{yield}
\SetKw{In}{in}
\SetKw{Not}{not}
\newcommand\mycommfont[1]{\small\ttfamily{#1}}
\SetCommentSty{mycommfont}

% PACKAGE SET UP -- Theorems, equations, definitions setup
\newtheoremstyle{plainspaced}
{1em} % Space above
{1em} % Space below
{\em} % Body font
{} % Indent amount
{\bfseries} % Theorem head font
{.} % Punctuation after theorem head
{.5em} % Space after theorem head
{\thmname{#1}\thmnumber{ #2}\thmnote{ (#3)}} % Theorem head spec (can be left empty, meaning `normal')

\newtheoremstyle{definitionspaced}
{1em} % Space above
{1em} % Space below
{} % Body font
{} % Indent amount
{\bfseries} % Theorem head font
{.} % Punctuation after theorem head
{.5em} % Space after theorem head
{\thmname{#1}\thmnumber{ #2}\thmnote{ (#3)}} % Theorem head spec (can be left empty, meaning `normal')

\theoremstyle{plainspaced}
\newtheorem{theorem}{Theorem}[chapter]			% Theorem style
\newtheorem{lemma}[theorem]{Lemma}				% Lemma style
\newtheorem{problem}[theorem]{Problem}			% Lemma style
\newtheorem{proposition}[theorem]{Proposition}	% Proposition style

\theoremstyle{plainspaced}
\newtheorem{definition}[theorem]{Definition}	% Definition style
\theoremstyle{definitionspaced}
\newtheorem{example}[theorem]{Example}			% Definition style

% Add symbols at the end of example and definitions
\AtEndEnvironment{example}{\null\hfill $\lrcorner$}%
\AtEndEnvironment{definition}{\null\hfill $\lrcorner$}%

% Must run this command to use index
\makeindex

% Spacing in easylist items and rows/cols in diagrams
\newcommand{\listSpace}{-0.25em}
\newcommand{\diagramSpace}{3em}

% Set up for Python code
\usepackage{xcolor}
\definecolor{maroon}{cmyk}{0, 0.87, 0.68, 0.32}
\definecolor{halfgray}{gray}{0.55}
\definecolor{ipython_frame}{RGB}{207, 207, 207}
\definecolor{ipython_bg}{RGB}{255, 255, 255}
\definecolor{ipython_red}{RGB}{186, 33, 33}
\definecolor{ipython_green}{RGB}{0, 128, 0}
\definecolor{ipython_cyan}{RGB}{64, 128, 128}
\definecolor{ipython_purple}{RGB}{170, 34, 255}

\lstset{
	breaklines=true,
	%
	extendedchars=true,
	literate=
	{á}{{\'a}}1 {é}{{\'e}}1 {í}{{\'i}}1 {ó}{{\'o}}1 {ú}{{\'u}}1
	{Á}{{\'A}}1 {É}{{\'E}}1 {Í}{{\'I}}1 {Ó}{{\'O}}1 {Ú}{{\'U}}1
	{à}{{\`a}}1 {è}{{\`e}}1 {ì}{{\`i}}1 {ò}{{\`o}}1 {ù}{{\`u}}1
	{À}{{\`A}}1 {È}{{\'E}}1 {Ì}{{\`I}}1 {Ò}{{\`O}}1 {Ù}{{\`U}}1
	{ä}{{\"a}}1 {ë}{{\"e}}1 {ï}{{\"i}}1 {ö}{{\"o}}1 {ü}{{\"u}}1
	{Ä}{{\"A}}1 {Ë}{{\"E}}1 {Ï}{{\"I}}1 {Ö}{{\"O}}1 {Ü}{{\"U}}1
	{â}{{\^a}}1 {ê}{{\^e}}1 {î}{{\^i}}1 {ô}{{\^o}}1 {û}{{\^u}}1
	{Â}{{\^A}}1 {Ê}{{\^E}}1 {Î}{{\^I}}1 {Ô}{{\^O}}1 {Û}{{\^U}}1
	{œ}{{\oe}}1 {Œ}{{\OE}}1 {æ}{{\ae}}1 {Æ}{{\AE}}1 {ß}{{\ss}}1
	{ç}{{\c c}}1 {Ç}{{\c C}}1 {ø}{{\o}}1 {å}{{\r a}}1 {Å}{{\r A}}1
	{€}{{\EUR}}1 {£}{{\pounds}}1
}

%%
%% Python definition (c) 1998 Michael Weber
%% Additional definitions (2013) Alexis Dimitriadis
%% modified by me (should not have empty lines)
%%
\lstdefinelanguage{iPython}{
	morekeywords={access,and,break,class,continue,def,del,elif,else,except,exec,finally,for,from,global,if,import,in,is,lambda,not,or,pass,print,raise,return,try,while},%
	%
	% Built-ins
	morekeywords=[2]{abs,all,any,basestring,bin,bool,bytearray,callable,chr,classmethod,cmp,compile,complex,delattr,dict,dir,divmod,enumerate,eval,execfile,file,filter,float,format,frozenset,getattr,globals,hasattr,hash,help,hex,id,input,int,isinstance,issubclass,iter,len,list,locals,long,map,max,memoryview,min,next,object,oct,open,ord,pow,property,range,raw_input,reduce,reload,repr,reversed,round,set,setattr,slice,sorted,staticmethod,str,sum,super,tuple,type,unichr,unicode,vars,xrange,zip,apply,buffer,coerce,intern},%
	%
	sensitive=true,%
	morecomment=[l]\#,%
	morestring=[b]',%
	morestring=[b]",%
	%
	morestring=[s]{'''}{'''},% used for documentation text (mulitiline strings)
	morestring=[s]{"""}{"""},% added by Philipp Matthias Hahn
	%
	morestring=[s]{r'}{'},% `raw' strings
	morestring=[s]{r"}{"},%
	morestring=[s]{r'''}{'''},%
	morestring=[s]{r"""}{"""},%
	morestring=[s]{u'}{'},% unicode strings
	morestring=[s]{u"}{"},%
	morestring=[s]{u'''}{'''},%
	morestring=[s]{u"""}{"""},%
	%
	% {replace}{replacement}{lenght of replace}
	% *{-}{-}{1} will not replace in comments and so on
	literate=
	{á}{{\'a}}1 {é}{{\'e}}1 {í}{{\'i}}1 {ó}{{\'o}}1 {ú}{{\'u}}1
	{Á}{{\'A}}1 {É}{{\'E}}1 {Í}{{\'I}}1 {Ó}{{\'O}}1 {Ú}{{\'U}}1
	{à}{{\`a}}1 {è}{{\`e}}1 {ì}{{\`i}}1 {ò}{{\`o}}1 {ù}{{\`u}}1
	{À}{{\`A}}1 {È}{{\'E}}1 {Ì}{{\`I}}1 {Ò}{{\`O}}1 {Ù}{{\`U}}1
	{ä}{{\"a}}1 {ë}{{\"e}}1 {ï}{{\"i}}1 {ö}{{\"o}}1 {ü}{{\"u}}1
	{Ä}{{\"A}}1 {Ë}{{\"E}}1 {Ï}{{\"I}}1 {Ö}{{\"O}}1 {Ü}{{\"U}}1
	{â}{{\^a}}1 {ê}{{\^e}}1 {î}{{\^i}}1 {ô}{{\^o}}1 {û}{{\^u}}1
	{Â}{{\^A}}1 {Ê}{{\^E}}1 {Î}{{\^I}}1 {Ô}{{\^O}}1 {Û}{{\^U}}1
	{œ}{{\oe}}1 {Œ}{{\OE}}1 {æ}{{\ae}}1 {Æ}{{\AE}}1 {ß}{{\ss}}1
	{ç}{{\c c}}1 {Ç}{{\c C}}1 {ø}{{\o}}1 {å}{{\r a}}1 {Å}{{\r A}}1
	{€}{{\EUR}}1 {£}{{\pounds}}1,
	%
	literate=
	*{+}{{{\color{ipython_purple}+}}}1
	{-}{{{\color{ipython_purple}-}}}1
	{*}{{{\color{ipython_purple}$^\ast$}}}1
	{/}{{{\color{ipython_purple}/}}}1
	{^}{{{\color{ipython_purple}\^{}}}}1
	{?}{{{\color{ipython_purple}?}}}1
	{!}{{{\color{ipython_purple}!}}}1
	{\%}{{{\color{ipython_purple}\%}}}1
	{<}{{{\color{ipython_purple}<}}}1
	{>}{{{\color{ipython_purple}>}}}1
	{|}{{{\color{ipython_purple}|}}}1
	{\&}{{{\color{ipython_purple}\&}}}1
	{~}{{{\color{ipython_purple}~}}}1
	%
	{==}{{{\color{ipython_purple}==}}}2
	{<=}{{{\color{ipython_purple}<=}}}2
	{>=}{{{\color{ipython_purple}>=}}}2
	%
	{+=}{{{+=}}}2
	{-=}{{{-=}}}2
	{*=}{{{$^\ast$=}}}2
	{/=}{{{/=}}}2,
	%
	%   identifierstyle=\color{red}\ttfamily,
	commentstyle=\color{ipython_cyan}\ttfamily,
	stringstyle=\color{ipython_red}\ttfamily,
	keepspaces=true,
	showspaces=false,
	showstringspaces=false,
	%
	rulecolor=\color{ipython_frame},
	frame=single,
	frameround={t}{t}{t}{t},
	xleftmargin=5.5ex,
	framexleftmargin=7mm,
	tabsize=4,
	numbers=left,
	numberstyle=\tiny\color{halfgray},
	%
	%
	backgroundcolor=\color{ipython_bg},
	%   extendedchars=true,
	basicstyle=\scriptsize\ttfamily,
	keywordstyle=\color{ipython_green}\ttfamily,
	escapechar=\¢,escapebegin=\color{ipython_green},
}


\lstdefinelanguage{iPythonLarge}{
	morekeywords={access,and,break,class,continue,def,del,elif,else,except,exec,finally,for,from,global,if,import,in,is,lambda,not,or,pass,print,raise,return,try,while},%
	%
	% Built-ins
	morekeywords=[2]{abs,all,any,basestring,bin,bool,bytearray,callable,chr,classmethod,cmp,compile,complex,delattr,dict,dir,divmod,enumerate,eval,execfile,file,filter,float,format,frozenset,getattr,globals,hasattr,hash,help,hex,id,input,int,isinstance,issubclass,iter,len,list,locals,long,map,max,memoryview,min,next,object,oct,open,ord,pow,property,range,raw_input,reduce,reload,repr,reversed,round,set,setattr,slice,sorted,staticmethod,str,sum,super,tuple,type,unichr,unicode,vars,xrange,zip,apply,buffer,coerce,intern},%
	%
	sensitive=true,%
	morecomment=[l]\#,%
	morestring=[b]',%
	morestring=[b]",%
	%
	morestring=[s]{'''}{'''},% used for documentation text (mulitiline strings)
	morestring=[s]{"""}{"""},% added by Philipp Matthias Hahn
	%
	morestring=[s]{r'}{'},% `raw' strings
	morestring=[s]{r"}{"},%
	morestring=[s]{r'''}{'''},%
	morestring=[s]{r"""}{"""},%
	morestring=[s]{u'}{'},% unicode strings
	morestring=[s]{u"}{"},%
	morestring=[s]{u'''}{'''},%
	morestring=[s]{u"""}{"""},%
	%
	% {replace}{replacement}{lenght of replace}
	% *{-}{-}{1} will not replace in comments and so on
	literate=
	{á}{{\'a}}1 {é}{{\'e}}1 {í}{{\'i}}1 {ó}{{\'o}}1 {ú}{{\'u}}1
	{Á}{{\'A}}1 {É}{{\'E}}1 {Í}{{\'I}}1 {Ó}{{\'O}}1 {Ú}{{\'U}}1
	{à}{{\`a}}1 {è}{{\`e}}1 {ì}{{\`i}}1 {ò}{{\`o}}1 {ù}{{\`u}}1
	{À}{{\`A}}1 {È}{{\'E}}1 {Ì}{{\`I}}1 {Ò}{{\`O}}1 {Ù}{{\`U}}1
	{ä}{{\"a}}1 {ë}{{\"e}}1 {ï}{{\"i}}1 {ö}{{\"o}}1 {ü}{{\"u}}1
	{Ä}{{\"A}}1 {Ë}{{\"E}}1 {Ï}{{\"I}}1 {Ö}{{\"O}}1 {Ü}{{\"U}}1
	{â}{{\^a}}1 {ê}{{\^e}}1 {î}{{\^i}}1 {ô}{{\^o}}1 {û}{{\^u}}1
	{Â}{{\^A}}1 {Ê}{{\^E}}1 {Î}{{\^I}}1 {Ô}{{\^O}}1 {Û}{{\^U}}1
	{œ}{{\oe}}1 {Œ}{{\OE}}1 {æ}{{\ae}}1 {Æ}{{\AE}}1 {ß}{{\ss}}1
	{ç}{{\c c}}1 {Ç}{{\c C}}1 {ø}{{\o}}1 {å}{{\r a}}1 {Å}{{\r A}}1
	{€}{{\EUR}}1 {£}{{\pounds}}1,
	%
	literate=
	*{+}{{{\color{ipython_purple}+}}}1
	{-}{{{\color{ipython_purple}-}}}1
	{*}{{{\color{ipython_purple}$^\ast$}}}1
	{/}{{{\color{ipython_purple}/}}}1
	{^}{{{\color{ipython_purple}\^{}}}}1
	{?}{{{\color{ipython_purple}?}}}1
	{!}{{{\color{ipython_purple}!}}}1
	{\%}{{{\color{ipython_purple}\%}}}1
	{<}{{{\color{ipython_purple}<}}}1
	{>}{{{\color{ipython_purple}>}}}1
	{|}{{{\color{ipython_purple}|}}}1
	{\&}{{{\color{ipython_purple}\&}}}1
	{~}{{{\color{ipython_purple}~}}}1
	%
	{==}{{{\color{ipython_purple}==}}}2
	{<=}{{{\color{ipython_purple}<=}}}2
	{>=}{{{\color{ipython_purple}>=}}}2
	%
	{+=}{{{+=}}}2
	{-=}{{{-=}}}2
	{*=}{{{$^\ast$=}}}2
	{/=}{{{/=}}}2,
	%
	%   identifierstyle=\color{red}\ttfamily,
	commentstyle=\color{ipython_cyan}\ttfamily,
	stringstyle=\color{ipython_red}\ttfamily,
	keepspaces=true,
	showspaces=false,
	showstringspaces=false,
	%
	rulecolor=\color{ipython_frame},
	frame=single,
	frameround={t}{t}{t}{t},
	xleftmargin=5.5ex,
	framexleftmargin=7mm,
	tabsize=4,
	numbers=left,
	numberstyle=\scriptsize\color{halfgray},
	%
	%
	backgroundcolor=\color{ipython_bg},
	%   extendedchars=true,
	basicstyle=\footnotesize\ttfamily,
	keywordstyle=\color{ipython_green}\ttfamily,
	escapechar=\¢,escapebegin=\color{ipython_green},
}

% -------------------------------------------------------------
% ---- MISC SET UP
% -------------------------------------------------------------

% Custom mathematics commands
\newcommand{\norm}[1]{\left\lVert#1\right\rVert}
\newcommand{\abs}[1]{\left\lvert#1\right\rvert}
\newcommand*\conj[1]{\overline{#1}}
\DeclarePairedDelimiterX{\innerprod}[2]{\langle}{\rangle}{#1, #2}
\DeclareMathOperator{\lcm}{lcm}
\DeclareMathOperator{\diag}{diag}
\DeclareMathOperator{\C}{\mathbb{C}}
\DeclareMathOperator{\R}{\mathbb{R}}
\DeclareMathOperator{\Z}{\mathbb{Z}}
\DeclareMathOperator{\conv}{\ast}
\DeclareMathOperator{\GL}{\operatorname{GL}}


% A command to add a blank page
\newcommand\blankpage{%
	\null
	\newpage}

% Header set up with fancyhdr
\lhead[\nouppercase{\leftmark}]{\thepage} % EVEN, ODD
\rhead[\thepage]{\nouppercase{\rightmark}} % EVEN, ODD
\cfoot[]{}

\pagestyle{fancy}
\fancypagestyle{normal}{%
	% Header set up with fancyhdr
	\lhead[\nouppercase{\leftmark}]{\thepage} % EVEN, ODD
	\rhead[\thepage]{\nouppercase{\rightmark}} % EVEN, ODD
	\cfoot[]{}}
\fancypagestyle{noheadername}{%
	% Header set up with fancyhdr
	\lhead[]{\thepage} % EVEN, ODD
	\rhead[\thepage]{} % EVEN, ODD
	\cfoot[]{}}

% Declare first page in every chapter as 'fancy' pagestyle
\makeatletter
\renewcommand\chapter{\if@openright\cleardoublepage\else\clearpage\fi
	\thispagestyle{fancy}%
	\global\@topnum\z@
	\@afterindentfalse
	\secdef\@chapter\@schapter}
\makeatother

% Custom environment for comments
\newenvironment{comment}{\begin{framed} \footnotesize \textcolor{red}{\textbf{Comment.}}}{\end{framed} \normalsize}

\begin{document}
	
	% ---------------------------------------------------------
	% ---- Title page
	% ---------------------------------------------------------
	
	\includepdf{cover/cover.pdf}
	\pagestyle{empty}
	\afterpage{\blankpage}
	\pagenumbering{roman}
	
	% ---------------------------------------------------------
	% ---- DOCUMENT INTRODUCTION
	% ---------------------------------------------------------
	
	\section*{Abstract}
	\pagestyle{normal}
	
	The abstract goes here.
	It should contain information about why the work was done, which problem that was solved, what was done, what is new, and how it was done.
	
	If you are curious about how to write mathematics, perhaps this is of help.
	
	\begin{easylist}[itemize]
		\ListProperties(Space=\listSpace, Space*=\listSpace)
		# \url{http://web.cs.ucdavis.edu/~amenta/w10/writingman.pdf}
		# \url{https://www.math.uh.edu/~tomforde/Books/Halmos-How-To-Write.pdf}
	\end{easylist}
	
	
	
	
	
%	\begin{comment}
%		From \cite{holst_scientific_nodate}
%		- Why did we do it? justify your work, put it into a broader context.
%		- what is the problem we are solving? (or which questions are answered)
%		- What did we do? list main results here.  take-home message.
%		- what is new? a brief comparison with state of the art.
%		- How did we do it? tools used, methods /theoretical tools
%	\end{comment}
%	
%	\begin{comment}
%		Guidelines from Nature
%		- 1-2 sentences with basic introduction to field, comprehensible to any scientist
%		- 2-3 sentences of more detailed background, comprehensible to scientists in related discliplines
%		- 1 sentence clearly stating the general problem
%		- 1 sentence summarizing the main result (``here we show...'')
%		- 2-3 sentences explaining main result in direct comparison to what was thought to be the case previously, or explain how main result adds to previous knowledge
%		- 1-2 sentences putting the result in general context
%		- 2-3 sentences to provide broader perspective
%	\end{comment}
%	
%	\begin{comment}
%		TODO: Write the abstract when everything else is finished.
%		See other theses on how to do this, or try the following:
%		(1) avoid notation if possible
%		(2) put the subject in context
%		(3) state results in non-technical language if possible
%		(4) mention technique if there is a new one
%		(5) never cite other papers in abstract.
%		see \url{https://www.maa.org/sites/default/files/pdf/students/Writing%20Abstracts.pdf}
%		or the book by Bodil
%	\end{comment}
	

	
	\vfill
	\section*{Acknowledgements}
	Write acknowledgements here.
	\vfill

	\setcounter{secnumdepth}{2}
	\setcounter{tocdepth}{1}
	\tableofcontents
	
	%\listoffigures
	
	%\listoftables
	
	\cleardoublepage
	\chapter*{Notation}
	\pagestyle{noheadername}
	\vspace*{-3em}
	%The following notation will be used throughout the thesis.
	\small 
	\begin{multicols}{2}
		\begin{easylist}[itemize]
			\ListProperties(Space*=\listSpace, Space=\listSpace, 
			Style1**=\bfseries, Style2*=, Hang=false, Progressive=1ex)
			
			# Groups
			## $\Z$ : Additive integers
			## $\Z_n = \Z / n \Z$ : Additive integers mod $n$
			## $\R$ : Additive reals
			## $T = \R / \Z$ : Additive reals mod $1$
			## $GL(n, \Z)$ : Invertible matrices over $\Z$

			# Objects
			## $G, H$ : Abelian groups
			## $U, V$ : Unimodular matrices
			## $I_n$ : Identity matrix of size $n$
			
			# Binary operators
			%## $\innerprod{\cdot}{\cdot}$ : Inner product
			## $+$ : Addition
			## $\conv$ : Convolution
			## $\oplus$ : Direct sum
			
			# Relations
			## $\cong$ : Isomorphic
			## $>$ : Greater than
			## $\geq$ : Greater than or equal to
			

				
		\end{easylist}
	\end{multicols}
	\normalsize
	
	\section*{Abbreviations}
	\vspace*{-0.25em}
		\begin{easylist}[itemize]
			\ListProperties(Space*=\listSpace, 
			Space=\listSpace, 
			Style1**=\bfseries, 
			Style2*=, 
			Hang=false, 
			Progressive=1ex)
			## ML - Maximum likelihood
			## MAP - Maximum a posteriori
		\end{easylist}


	\cleardoublepage
	
	% ---------------------------------------------------------
	% ---- MAIN PART OF DOCUMENT
	% ---------------------------------------------------------
	
	\pagenumbering{arabic}
	\pagestyle{fancy}
	
	%\chapter{Notes}
	%\input{chapters/0_notes}
	
	\pagestyle{normal}
	\chapter{Introduction}
	\section{Introduction}
% write 1-2 pages -- brief history, overview of the field, what the thesis is about, what the goal of the thesis is, and what is new.
The introduction goes here.

\section{Chapter overview}

An overview of each chapter goes here.

\paragraph{Chapter 1 -- Introduction}
This chapter.

\paragraph{Chapter 2 -- Basic \LaTeX{}}
Some latex basics here.

\paragraph{Chapter 3 -- More \LaTeX{}}
Even more latex stuff here.

\paragraph{Appendices}
There is only one appendix, namely Appendix \ref{appendix:appendix_A}, which contains an example of how Python code can be included in a thesis.

	\chapter{Basic \LaTeX{}}
	\label{chapter:basic_latex}

% Paragraph descriping what the chapter is about
\blindtext

\section{A section with a definition and a theorem}
\label{sec:a_section_with_a_definition_and_a_theorem}

The above is a section. Here's a definition.

\begin{definition}[Conditional probability]
	\label{def:conditional_probability}
	The conditional probability of $A$ given $B$ is defined as
	\begin{equation*}
		P(A | B) = \frac{P(A \cap B)}{P(B)},
	\end{equation*}
	where $A \cap B$ denotes the intersection of $A$ and $B$.
\end{definition}

\begin{theorem}[Bayes theorem]
	Given $P(A | B)$, $P(A)$ and $P(B)$, we can compute $P(B | A)$ using 
	\begin{equation}
	\label{eqn:bayes_theorem}
	P(B | A) = \frac{P(A | B) P(B)}{P(A)}.
	\end{equation}
\end{theorem}
\begin{proof}
	Write $P(A \cap B)$ in two ways using the Definition \ref{def:conditional_probability} of conditional probability as follows.
	\begin{equation*}
		P(A \cap B) = P(A | B) P(B) \qquad P(B \cap A) = P(B | A) P(A)
	\end{equation*}
	The intersection is symmetric, meaning that $B \cap A = B \cap A$.
	Thus we can compare terms and write $P(A | B) P(B) = P(B | A) P(A)$, rearranging this gives Bayes theorem.
\end{proof}

Bayes theorem has many applications, such as the \emph{Naive Bayes Classifier}, which is a machine learning algorithm.
The classifier assigns a label to a piece of data, e.g. classifying an email as spam or not.
It's called ``naive'' since it assumes conditional independence.
Equation \eqref{eqn:bayes_theorem} has extensions when more variables are used.

\section{A section with an example}
In Section \ref{sec:a_section_with_a_definition_and_a_theorem} we gave a theorem, here's an example with a real world application.

\begin{example}[An example with numbers]
	Here's a little example with some numbers.
	\begin{align*}
		P(B | A) &= \frac{P(A | B) P(B)}{P(A)} \\
			 	 &= \frac{(0.3) (0.4)}{0.24} = \frac{0.12}{0.24} = \frac{1}{2}
	\end{align*}
	As you see, applications are everywhere.
\end{example}

\blindtext

\section{A section with a figure}
\blindtext
	\begin{figure}[ht!]
	\centering
	\includegraphics[width=0.3\linewidth]{figs/truncatedoctahedron}
	\caption{A pretty figure.}
	\label{fig:truncatedoctahedron}
	\end{figure}
\blindtext



	
	\chapter{Some more \LaTeX{}}
	\label{chapter:some_more_latex}

\section{References and lists}

Although \cite{strang_linear_1976} is a great introduction to linear algebra, \cite{roman_advanced_2005} presents the material in a more abstract way.

Here's a list of some linear algebra concepts.
\begin{easylist}[enumerate]
	\ListProperties(Space=\listSpace, Space*=\listSpace, Numbers1=a, Numbers2=r)
	
	# Linear algebra is about matrices and vectors.
	
	## The elements are usually numbers from $\R$ or $\C$.
	## Matrices are given by two indices, vectors by one.
	
	# Matrix multiplication is given by $y_i = \sum_j A_{ij} x_j$.
	
	# If $A = a_{ij}$, then the tranpose flips across the diagonal so that $A^T = a_{ji}$.
	The transpose is the adjoint, i.e. $\innerprod{Ax}{y} = \innerprod{x}{A^Ty}$.
\end{easylist}


Here's a different type of list.

\begin{easylist}[enumerate]
	\ListProperties(Space=\listSpace, Space*=\listSpace, Numbers=r, FinalMark={)})
	
	# Linear algebra is about matrices and vectors.
	
	## The elements are usually numbers from $\R$ or $\C$.
	## Matrices are given by two indices, vectors by one.

\end{easylist}

Here's a different type of list.

\begin{easylist}[itemize]
	\ListProperties(Space=\listSpace, Space*=\listSpace)
	
	# Linear algebra is about matrices and vectors.
	
	## The elements are usually numbers from $\R$ or $\C$.
	## Matrices are given by two indices, vectors by one.
	
	# Matrix multiplication is given by $y_i = \sum_j A_{ij} x_j$.
	
	# If $A = a_{ij}$, then the tranpose flips across the diagonal so that $A^T = a_{ji}$.
	The transpose is the adjoint, i.e. $\innerprod{Ax}{y} = \innerprod{x}{A^Ty}$.
\end{easylist}


\section{Algorithms}

Here's an algorithm.

\begin{algorithm}
	\label{algo:student_algorithm}
	\SetKwInOut{Input}{Input}
	\SetKwInOut{Output}{Output}
	%\DontPrintSemicolon % \PrintSemicolon

	\Input{Student $s$.}
	\Output{Inverse $s^{-1}$ such that $s \circ s^{-1} = s^{-1} \circ s = \text{Id}$.}
	\BlankLine
	
	\tcp{Morning.}
	Grab coffee\;
	\While{thesis not finished}{
		Read mathematics\;
		Write thesis\;
	}
	Relax for an hour\;
	\BlankLine
	\tcp{Evening.}
	\For{every friend $f \in F$}{
		Call $f$\tcp*{Give your friends a call.} \label{algline:call_friends}
		}
	Sleep\;
	\caption{Algorithm for a thesis.}
\end{algorithm}

Notice in Algorithm \ref{algo:student_algorithm} above that the student does in fact call his friends in line \ref{algline:call_friends}.

Here's some Python code.

\lstinputlisting[language = iPython, firstline=9]{./code/example_code/composition.py}


\section{title}
\blindtext

\subsection{title}
\blindtext

\subsubsection{title}
\blindtext

\subsubsection{title}
\blindtext

\section{title}
\blindtext
\subsection{title}
\blindtext

\subsubsection{title}
\blindtext

	
	% ---------------------------------------------------------
	% ---- INDEX
	% ---------------------------------------------------------
	
	%\printindex
	
	% ---------------------------------------------------------
	% ---- BIBLIOGRAPHY
	% ---------------------------------------------------------
	
	\bibliographystyle{apalike}%alpha, apalike is also good
	\bibliography{thesis_bibliography}
	
	% ---------------------------------------------------------
	% ---- APPENDICES
	% ---------------------------------------------------------
	
	\begin{appendices}
		\chapter{Example code}
		\label{appendix:appendix_A}

sdf



	\end{appendices}

\end{document}